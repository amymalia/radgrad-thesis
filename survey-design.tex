\chapter{Survey Design}
\label{surveyDesign}
The first part of the experimental design of this research project involves a baseline survey. The main goal of this survey is to gain better and more specific insight into the current undergraduate experience. This survey was given to current undergraduate ICS students and prospective ICS students. It was deployed electronically via Google Forms and students completed the survey on an iPad either before or after an advising session with an ICS advisor.  The original goal was to gather at least 50 responses. Ultimately, the results of this survey may be useful in a future study that compares the ICS student experience before and after integrating RadGrad. 
\section{Baseline Survey}
\label{baselineSurvey}
	The baseline survey contains the following questions. The full assessment can be found in Appendix A. 

\subsection{Basic Information}
\begin{enumerate}
\item \textit{What is your gender? }
Goal: Since the ICS program currently has significantly more male students than female students, what are are the differences between the experiences of the two genders? Could this give any insight into why there are so little female students? Is this something RadGrad could address? After implementing RadGrad, have there been any differences in the gender ratio or the disparity between the experiences of the two genders? Ideally, both genders should have equally positive experiences in the ICS program.
\item \textit{What is your current status in the ICS degree program?}
Goals: How do student experiences evolve as they progress through the ICS degree program? Are there any patterns? Future goals: Does RadGrad have any effect on this? Ideally, students from all levels should have equally positive experiences in the ICS program. 
\end{enumerate}

\subsection{Prospective ICS Students}
\begin{enumerate}
\item\textit{ How EXCITED are you about entering the ICS program? Rank from 1-5.}
Goals: This will provide information regarding how students view the ICS department, based solely on outside information and before their own experiences. 
Future goals: Compare this answer to the same question on the post-deployment assessment. Ideally, RadGrad will create more excitement among prospective students due to better presentation and the appearance of a strong, supportive community and satisfied alumni.
\item \textit{How INTIMIDATED do you feel about entering the ICS program? Rank from 1-5.}
Goals: This will provide information regarding how students view the ICS department, based solely on outside information and before their own experiences.
Future goals: Compare this answer to the same question on the post-deployment assessment. Ideally, RadGrad will create less intimidation among prospective students due to the appearance of a strong, supportive, and diverse community and satisfied alumni. 
\end{enumerate}

\subsection{Current ICS Students}
\begin{enumerate}
\item \textit{Which of the following extracurricular activities, if any, pertain to you? }
Goals: This will provide information about how much initiative students are currently taking to get additional ICS education and experience outside of the classroom.
Future goals: Compare this answer to the same question on the post-deployment assessment. Ideally, RadGrad will increase the amount of student involvement in outside ICS-related activities due to providing students with stronger connections to the ICS community.
\item \textit{Do you feel that you get enough support from others in the ICS department?}
Goals: Are students lacking support in certain areas? If so, how can RadGrad help to address this? 
Future goals: Compare this answer to the same question on the post-deployment assessment. Ideally, RadGrad will provide a way to give more students the support they desire from others in the department. 
\item \textit{As a student, do you feel like you have a voice to make changes within the department?}
Goals: If most students indicate that they do not feel like they have a voice within the department, what can RadGrad do to address this problem?
Future goals: Compare this answer to the same question on the post-deployment assessment. Ideally, RadGrad will cause more students to feel like they do have a voice to make changes in the department.
\item \textit{What makes you proud to be a part of the ICS department?}
Goals: This provides information about how current students view the department. A successful department should have a positive reputation among students, which can be manifested with a sense of pride. 
Future goals: Compare this answer to the same question on the post-deployment assessment. Ideally, RadGrad will cause positive changes in the ICS department's reputation, leading to a greater sense of pride among students, which may play a role in students' success
\end{enumerate}

\subsection{Current ICS Students: Influences}
\begin{enumerate}
\item \textit{To what extent have ICS alumni influenced your development in the ICS program?}
Goals: This provides information about the extent of academic and professional interaction between ICS students and alumni.
Future goals: Compare this answer to the same question on the post-deployment assessment. Ideally, RadGrad will facilitate more student-alumni interaction, and cause more students to be influenced in some way by an alumn in an ICS-related way.
\item \textit{To what extent have ICS peers influenced your development in the ICS program?}
Goals: This provides information about the extent of academic and professional interaction between ICS students and their peers.
Future goals: Compare this answer to the same question on the post-deployment assessment. Ideally, RadGrad will facilitate more peer interaction, and cause more students to be influenced in some way by a peer in an ICS-related way.
\item \textit{To what extent have you influenced your ICS peers’ development in the ICS program?}
Goals: This provides information about how students perceive their academic and professional interactions with their peers.
Future goals: Compare this answer to the same question on the post-deployment assessment. Ideally, RadGrad will facilitate more peer interaction, and cause more students play a role in influencing their peers in an ICS-related way.
\end{enumerate}

\subsection{Graduating ICS Students}
\begin{enumerate}
\item \textit{Now that you are nearing the end of your ICS degree program experience, how well prepared do you feel to find a job after graduation?}
Goal: If the ICS department is fulfilling its duty, most graduating students should feel at least adequately prepared (ideally well prepared) for the future. If most students indicate that they do not feel prepared, what can RadGrad do to address this problem? 
Future goals: Ideally, after deploying RadGrad, a higher percentage of students will feel either adequately prepared or well prepared for the future.
\item \textit{If you answered above that you feel unprepared to find a job after graduation, please explain why. }
Goal: Are there any common reasons for students not feeling prepared? If so, is there anything RadGrad can do to address this problem? 
Future goals: Ideally, after deploying RadGrad, there will be a lower percentage of students who indicate the same problems as the preliminary questionnaire. 
\end{enumerate}

\section{Analysis}
After collecting 100 survey results, I analyzed the data to see if there are any notable patterns, or any indications of common problems that could be addressed by RadGrad. Ideally, after RadGrad, future studies will show that there is less disparity between student expectations and reality, greater student satisfaction with the department, more student engagement, and more positive student feelings overall.
