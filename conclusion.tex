\chapter{Conclusions}
The main goal of this thesis is to find evidence of a problem, gather more concrete evidence of this problem, and attempt to solve the problem with the initial design and implementation of a system. When I initially gathered TechHui data about the pros and cons of being an ICS student at UH Manoa, I found the first evidence of a problem: over the past eight years, students were not fully satisfied with the ICS experience. I then designed and implemented an ICS experience survey, which surveyed 100 current ICS students and asked more specific questions about their ICS experience. The results of this survey gave more concrete evidence that there is room for improvement in the ICS department when it comes to encouraging and enabling students to participate in extracurricular activities, giving students the support they need from all members of the department, and encouraging and enabling students to interact with each other and ICS alumni on an academic or professional level. Along with the rest of the RadGrad team, I have helped to design and implement a system called RadGrad, which aims to address many perceived deficiencies within the ICS department. By combining degree planning, social networking, and gamification, RadGrad aims to improve the ICS student experience on academic, professional, and social levels. 

After completing this thesis, RadGrad development will continue, and is scheduled to be deployed within the ICS department in Fall 2017. Future studies will be necessary to test whether or not RadGrad is adequately addressing problems within the department. After students have integrated RadGrad into their life for at least one semester (the time needed to go through once registration process and have enough time to participate in opportunities and courses), future studies may want to conduct another survey with ICS students. This survey could include similar questions to the survey conducted in my survey, which can then be used to compare and contrast pre- and post-RadGrad results. This survey could also include more RadGrad specific questions, to get an idea of how students feel about using the system. Furthermore, gathering usage statistics could possibly add valuable insight into how users are actually responding to and interacting with the system. Based on the results of these studies, RadGrad could either be further improved to better solve the perceived problems, or discarded if there is no evidence that RadGrad has any positive impact on the ICS community.

Assuming that RadGrad is successful within the ICS department, future possible expansions include integration into other departments at UH Manoa, being established as a staple UH system that will get its own funding and staff positions, being integrated at other universities, and finally being integrated in other environments, such as within tech companies. 

This thesis marks the beginning of the RadGrad journey, and will hopefully be the first of many studies. After completing this thesis, overall, the results suggest that RadGrad is progressing in a promising direction, and if it continues on that path, it has the potential to change the lives of future students drastically.   