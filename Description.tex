\chapter{Description}
I believe that the best way to address current ICS student issues is through an online system that combines degree planning, social networking, and gamification. In this chapter I present the design of such a system. 
\section{Degree Planner}
\subsection{Degree Plan}
Each student in RadGrad will have a degree plan, which displays the student's courses, extracurricular activities, and outside work on a semester-by-semester basis. This plan contains future data as well as historical data. This allows students to easily view their progress and prepare for the future. 
\subsection{Degree Goal}
Although each student is aiming for a bachelor's degree in ICS, a more specific goal is beneficial in helping the student find a focus for their education and career goals. Some of these specific goals include B.S./B.A/B.S Computer Engineering and Security/Ph.D. Prep/Silicon Valley Tech. By specifying these specific goals in a concrete way, students can feel less overwhelmed by the large expanse of ICS classes, more prepared for the future, and more easily form communities of interest with other like-minded students.
\subsection{Dashboard}
Each user has access to a personal dashboard, available upon login. This dashboard provides a quick look at some of the user's stats, such as current ICS GPA, current ICS credits awarded, summary of schedule, current degree goals, interest tags, user picture, suggested vignettes, stoplight, recommendations, currently active petitions, and predictions. This is a quick and easy way to provide students with a variety of information on their overall progress in their major. 
\subsection{Predictions}
Each student will have a prediction model, which predicts post-graduation aspects based upon recent graduates, data from local tech organizations, recruiters, headhunters, etc., and data from the ICS faculty. This data is then combined with the student's individual degree plan and degree goals to produce a customized prediction.  This feature will hopefully help students feel better prepared for their future after graduation.

\subsection{Recommendations}
Recommendations aim to help students understand how to change their current behavior to improve their ICS experience. Unlike Starfish by Hobsons, RadGrad will base recommendations on a factors beyond academics, such as the student's current degree plan, degree goals, and professional interests. Some examples of possible recommendations are: relevant courses or extracurricular activities not already present in their degree plan, an estimate of ideal maximum work hours, predicted impact of their GPA, and relevant mentorship opportunities that are not being taken advantage of. 
\subsection{STAR Interface}
RadGrad will request a relationship with the UH STAR website, which will provide students with their current and past courses and their resulting grades. This will allow students to see all of their relevant course information in the same place that they get all their other ICS information. RadGrad could also integrate with the STAR scholarship database to find ICS-related scholarships and encourage students to apply for them, based off their information and goals. For instance, if one of the student's main goals is to graduate debt-free, they should be informed of applicable scholarships. In addition to the information provided by STAR, since the database will be much smaller, RadGrad could provide more specific details about each scholarship, how to become eligible, and how to apply. 
\subsection{Workload Adviser}
RadGrad will implement a virtual workload adviser, which will combine the student's course load, outside work hours, ICS grade data, and employer expectations to give the student advice on how much work they should be taking on at one time. It will offer suggestions, such as ``If you drop 1 ICS course and reduce your work hours to 10 per week, the average ICS GPA is 3.4." Unlike Blackboard Planner, RadGrad offers suggestions for students in the present time, rather than just for the future. This can help students to have a more realistic view of their goals, prevent students from getting burnt out, reduce stress levels, and encourage a healthier school-work balance.

\section{Social Network}
\subsection{Profile}
Students, faculty, graduates, and administrators will each have their own respective profiles. Student profiles will contain personal information including name, email, details about their degree, images, interests, projected graduation date, and professional recommendations based off their inputted information. Faculty profiles include name, email, image, professional interests, and descriptions of current projects they are working on. Graduate profiles will include details about their life after graduation such as place of employment and position description. All profiles will be publicly available so that the ICS community can view each others' profiles and find connections. 

On a gamification level, profiles also allow users to be persistent and easily view their progress. Since each user has a username and profile, they will be persistent on RadGrad, and all of their achievements will build up on their profile as they progress through college and beyond.
\subsection{Mentorship}
RadGrad users can be certified mentors if they are in ICS 390 or a TA. A student is working under a mentor if they are a ICS 499 student working under a professor or participating in a research group opportunity under a faculty sponsor. There can be other possible instances of mentorship, and over time, a network of mentorships will form, which could possibly help foster more and better mentorships with future students.
\subsection{Department Feedback}
Any type of user can initiate department feedback by starting a petition. The petition will be public, and any other user can edit it as well. Once the editors of the petition reach a consensus and get at least 20 votes of confidence from other users, the user will become finalized and other users may sign the petition over a course of two weeks. The petition will then be discussed at a faculty meeting, eventually leading to the petition being implemented, not implemented, or deferred. By giving users a platform to easily collaborate with each other over a common cause, members of the ICS department will feel more empowered, more involved, and ideally more satisfied.
\subsection{Course Feedback}
As a supplement to the UH system's end-of-semester course feedback system, RadGrad offers a mid-semester public course evaluation system. This allows students to reach out and communicate with each other and professors to make the learning experience as ideal as possible for all parties, while there is still time left in the semester. Ideally this will improve the quality of courses, the satisfaction of the students, and the teaching abilities of the professors. Unlike Rate My Professor, this process is not anonymous, and encourages professors to improve, rather than encouraging other students to avoid a certain course. 
\subsection{Degree Feedback}
RadGrad will reach out to ICS alumni approximately six months after graduation. At this time, these ICS graduates will answer a number of questions regarding their life after graduation (i.e. graduate education, career prospects, retrospective thoughts about the ICS department, etc.). This feature will provide data that will help the ICS department improve their degree program, and also provide clear and convenient lines of communication between alumni and the ICS department. 
\subsection{Feedback and User Evaluation}
RadGrad will provide a easy and convenient way for users to provide RadGrad feedback about the system. There will be links for immediate feedback, and there will also be yearly surveys to measure user satisfaction over time. By providing an easy stream of communication between users and RadGrad developers, RadGrad can be constantly growing along with the department in order to serve their evolving needs as best as possible.
\subsection{Billboard}
RadGrad can also provide some physical hardware (i.e. a large monitor) to be displayed in the ICS department. This display will show aspects of RadGrad (i.e. statistics gathered from RadGrad, upcoming events, current petitions, etc.). This will further encourage engagement throughout the department, as it will be a constant reminder of the current status of the community, and will perhaps contribute to promoting an overall closer ``community" feel. 

\section{Gamification}
\subsection{Stoplight}
The stoplight is a UI widget embedded in the dashboard, which takes on the appearance of a traffic stoplight, and uses the red/yellow/green colors to indicate the extent of that student's ICS activity. The light is green if the student is taking excellent advantage of what the department has to offer. The light is yellow if the student is taking sufficient advantage of what the department has to offer, and the light is red if the student is not taking enough advantage of what the department has to offer. To determine this, the stoplight takes into account the student's professional interests and goals, their GPA, available opportunities, the opportunities they have taken advantage of, the available courses, and the courses that the student has taken. By being encouraged by the changing colors of this stoplight, fueled solely by the student's actions, students may take this as a personal challenge, or game, to keep the stoplight at a certain color as much as possible. This is a simple way for students to track their progress throughout their ICS journey. This is an example of small rewards that are given throughout the "game" of ICS--students can visually see improvement, as though they are achieving a new rank. Similarly to EXP points in games, students can get these achievements simply by putting in more time and effort. They don't necessarily have to win a Hackathon or get straight A's, but as long as they are participating and putting in the effort, it will show on their Stoplight. 

\subsection{Leaderboard}
A public leaderboard will be available for students to actively compare themselves against others in the department in terms of ICS GPA, velocity (a calculated value indicating a student's progress through the program), and professional preparation (a calculated value combining coursework and extracurricular activities). An award ceremony type of tradition could be started out of this, which awards high ranking students. This type of active and constant comparison could help foster healthy competition and higher engagement among students. This is an example of the multiplayer aspect of the aforementioned games that foster healthy competition and drive within players. Without the RadGrad network, students are left mostly oblivious to the progression of others and only have themselves to relate their progress to. If a student has always progressed slowly, that student may believe that he/she is continuing to do well, as long as they do not begin to progress even slower. With RadGrad, this student would be able to see the quicker progression of other students, and suddenly create a different goal in his/her mind.

\subsection{ICE}
ICE is an acronym for Innovation (i.e. a student's involvement in research or other innovative activity), Competency (i.e. a student's grades in ICS courses), and Experience (i.e. a student's involvement in high tech environments through internships or other professional activities). ICE is a measurement of these aspects, calculated using the information provided on the student's profile. This balances the three aspects to emphasize the importance of all three in an ideal ICS experience. Details on how these aspects are measured can be found at www.radgrad.org. These clear measurements of ``success" can be eye opening for students. It can be easy to get caught up in the minor details of a major and lose track of the bigger picture. ICE helps students to remind themselves to balance their ICS experiences out, in order to become as attractive as possible for future employers. This feature can also be physically manifested in terms of badges or stickers, as a symbol of rank. This can encourage students to become more competitive and therefore less lazy and more productive. (It helps that a lot of ICS students enjoy the incentives provided by video games.) This is also an example of the rewards and competitive multiplayer aspect of many popular video games. 




