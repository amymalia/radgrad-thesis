\chapter{Introduction}
\label{introduction}

Getting a college degree is an investment that many people make, with the expectation that once they get their degree, they will have the basic skills needed to start a career in the field of their choice. However, what happens if a student does everything they are told to do (takes required courses, gets a high GPA, and graduates on time), but once they start applying for jobs, they are constantly turned down and told that their coursework and high GPA are not enough? If colleges promise to prepare students for the workforce, they should be doing everything they can do deliver this promise, even if it means redefining what it means for a student to be "successful."   

For ICS students in 2017, it is hard to land a job when the only thing on your resume is a couple of standard programming courses and 3.8 GPA. Several ICS alumni have told me that they realized too late that employers are scouring incoming resumes for other things like internships, independent projects, and hackathons. It is also difficult for ICS students to find jobs in new industries that may not have existed, or may have been far less prominent four years ago when they began their ICS degree. How can these students prepare for these new fields, when they had no concept of it during their degree experience? Furthermore, is it reasonable to expect colleges to create new courses each time there is an advancement in technology?

These initial observations made me wonder if other ICS students over past decade were experiencing similar problems. To answer this question, I navigated to the Hawaii technology community site, TechHui \cite {TechHuiQuestions}, and found a forum question entitled "Three bad things about being an ICS student." I gathered responses from 199 ICS students from 2008 to 2016, and compiled a list of the ten most common complaints over the past 8 years:

\begin{enumerate}
  \item The ICS department needs to offer classes more frequently.
  \item The ICS department needs to offer a wider variety of classes.
  \item The ICS department needs a better sense of community.
  \item Some of the professors in the ICS department need to improve their teaching.
  \item The ICS department should offer more focused areas of study.
  \item ICS classes are too time consuming and take up more time than anticipated.
  \item The ICS department should offer more classes that meet focus requirements.
  \item ICS books are too expensive.
  \item ICS courses should involve more group work 
  \item ICS should encourage more interaction among students.
\end{enumerate}

Complaints 1, 2, 5, 6, 7, and 8 suggest problems with degree planning and coursework itself and complaints 3, 4, 9, and 10 suggest social and communication related problems within the department. There were also some other complaints among students on TechHui that were not as common but stuck out to me nonetheless. There were at least eight students who mentioned that they felt intimidated when they started out in ICS, due to the impressions they got from their classmates and the major overall. This caused them to feel discouraged and had an overall negative impact on their ICS experience. These sentiments further suggest social problems with the ICS community, as well as with how the department is perceived outside of the community. 

As ideal as it would be, it is hard to meet the needs of all current, past and present students in a department. However, after taking student and alumni feedback into consideration, several of these problems could potentially be alleviated by creating an online platform that provides students with the help they need to become a truly successful student--academically, professionally, and socially. Creating a useful degree planner that helps students get all the information they need to create an ideal plan for their personal goals could help students academically. Creating an online social network for the ICS community could help encourage students to connect with others in the department. Adding gamification aspects to the degree experience could help give students extra incentives to go beyond the UHM graduation requirements, and take the initiative to become more well prepared for finding a job after graduation. By combining these three components (degree planner, social network, and gamification), a new system called RadGrad could potentially address many of the aforementioned student problems and needs.  

