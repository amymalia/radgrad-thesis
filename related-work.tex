\chapter{Related Work}
RadGrad can be reduced into three major parts: degree planner, social network, and gamification. All three of these parts are combined to create a robust, interactive, and effective system to enhance the academic journeys of current and future ICS students. 
\section{Degree Planners}
\subsection{STAR}
STAR is the degree planning system currently used by the University of Hawaii. As of September 2016, the student interface provides five main capabilities: Academic Essentials, Graduation Pathway, What If Journey, Transcripts, and Scholarships. 

\subsubsection{Academic Essentials}
This interface provides information about the student's academic progress, and compares it to the student's academic requirements to show how close the student currently is to graduation. This information includes credit totals, grades, and required courses. This interface also includes a section for ``Advisor Notes", which is filled out during advising sessions. There is another section for ``Events and Actions" which lists important student academic events such as college applications, admittance, and graduation, and student academic actions such as Deans List award. A third section is called ``Educational Goals", which provides the student's ``immediate goals" and ``highest ed goals" on a semester-by-semester basis. This information is provided by the student through occasional assessments upon log-in to STAR. The top of the page also has a section for students with financial aid.
\subsubsection{Graduation Pathway}
This interface is provided for certain programs or exploratory or pre-major students. It shows the course information for the courses that the student has taken previously and is currently enrolled in, and shows which requirements each course fulfills. It also shows future semesters and suggests future types of classes that the student should enroll in, in order to fulfill their major requirements. This interface does not suggest specific classes, but only lists the requirement that the class will need to fulfill. 
\subsubsection{What If Journey}
This interface is provided for undergraduates at UH Manoa. It allows students to choose a different major than the one they are currently in. The page then reloads to show the STAR homepage, altered to show the requirements of the chosen major. This shows students where they would be in the program if they were to switch majors.
\subsubsection{Transcripts}
This interface allows students to access their campus transcripts by semester and by department. It also allows transfer students to access their transfer transcripts by semester and by institution. 
\subsubsection{Scholarships}
This interface allows students to find scholarships by either using a keyword search or by selecting the ``My Best Fit Scholarship" tab, which presumably gathers student academic data and compares it with scholarship data to find matches.

\subsection{MyEdu} 
The MyEdu website states that it aims to help students succeed in college, tell their stories, and get jobs and internships. It is free to sign up, and the website reports that ``twice as many MyEdu users graduate on time, compared to the national average." There are seven main capabilities of this website: GPA Calculator, Schedule Planner, Profiles, Internships/Jobs, Professor Recommendations, Course Grades, and Degree Timeline.
\subsubsection{GPA Calculator}
Students can input grades and course information to calculate their course GPA, semester GPA, and overall GPA. The calculator allows students to define grading styles and scales for each course, and set grade totals as well, which can assist students in figuring out minimum grades needed for desired final grades. The GPA calculator also comes with an Assignment Tracker, which allows students to input their assignments, due dates, and outcomes. This feature can also email automatic reminders to students regarding due dates. Finally, there is also a questions \& answers section which allows students to interact with other students in the same major. 
\subsubsection{Schedule Planner}
Students can create class schedules based on professor recommendations, grade distribution, and the time of the day. Schedules can be easily shared among classmates and study groups. There are also class schedule generator features, which includes an auto scheduler that automatically plans a student's schedule given professor recommendations, overall GPA, preferred days of the week, and preferred times of day.  When testing out this feature, not all ICS classes were listed, and I was only allowed to choose the ones listed. After submitted one class, I got an nondescript error and was not allowed to proceed.
\subsubsection{Profiles}
Students can create a personalized profile which showcases accomplishments, projects, volunteer-ships, internships, and work experiences. These profiles are public to other members, encouraging students to make their profile stand out from the crowd. Student information is displayed in a Windows 8-esque type of block format. 
\subsubsection{Internships/Jobs}
MyEdu includes a job search engine boasting ``over 2,000,000 entry level job listings." Users can search through the database using location, keywords, and job type. Users can save jobs to review later and save jobs they plan to apply for to keep their job search organized. 
\subsubsection{Professor Recommendations}
Professor recommendations are integrated within other MyEdu applications. These recommendations include study tips and exam types, lecture and attendance policy, teaching style and effectiveness, and official school evaluations. MyEdu prides their recommendation system on being superior to other popular recommendation systems (i.e. RateMyProfessor) due to including more relevant rating categories, as opposed to purportedly less relevant categories such as ``hotness". 
\subsubsection{Course Grades}
MyEdu works with universities to incorporate official course grade records. However, instead of just showing the students their personal records, MyEdu takes it up a notch by also displaying all encompassing statistics, such as class graduate distribution, average course GPAs, and average drop rates. These types of data can assist students in making more educated decisions in the future when they sign up for courses. Similar to my experience with the Schedule Planner, when testing out this feature, not all ICS classes were listed, and I was only allowed to choose the ones listed. After submitted one class, I got a server error and was not allowed to proceed. 
\subsubsection{Degree Timeline}
This feature shows an overall view of the student's progress in graphical form. Similar to STAR's academic essentials, it displays course totals, credit totals, and graduation progress percentages. MyEdu stresses that this feature allows students to avoid the ``senior surprise."

\subsection{Starfish by Hobsons}
The slogan for Hobsons is ``Education Advances: Imagine a world where all students find their best fit." Hobsons offers a wide range of educational solutions, ranging from students K-12 to college students. Starfish by Hobsons is one of their platforms which focuses on success, support, and retention initiatives, and engaging students more effectively with the campus community. There are three main parts of the Starfish Enterprise Success System: Early Alert, Connect, and Degree Planner.
\subsubsection{Early Alert}
Early Alert is a early warning and student tracking model which mines student performance data from existing technologies at the particular institution to detect at-risk students. These students are detected early enough, such as at the first sign of a problem, so that there is enough time to make a difference. There is a type of reward system called Kudo (a positive feedback note), which is used to encourage students and reward them for improvement or good work. 
\subsubsection{Connect}
Connect is an online appointment scheduling and case management system. This system promotes communication between students and their advisers, instructors, and tutors by means of in person meetings, phone calls, or virtual meetings. Connect includes a kiosk to allow easily scheduled walk-in meetings. These kiosks can help staff to manage a student queue and also allows students to check wait times remotely, which can save a lot of time and frustration. Connect also includes a road map for each student, which documents the steps a student must take to achieve his or her goals. This map is created by an adviser and is visible to all members of the student's support network. 
\subsubsection{Degree Planner}
Degree Planner provides academic templates which advisers can use to easily edit to adjust to a particular student's needs. It also focuses on students' constantly changing goals and ability to adjust the student's plan to accommodate these goals. When a student deviates from their given plan, the student's adviser is notified so that they can plan a meeting with the student to check on their status and re-identify their goals. 

\subsection{College Scheduler}
The College Scheduler company has two products: Schedule Planner and Pathway Planner. The Schedule Planner focuses on optimizing the way students can plan their schedules, and the Pathway Planner focuses on optimizing the way students progress towards graduation.
\subsubsection{Schedule Planner}
Schedule Planner allows students to easily schedule (or automatically generate) their classes around outside obligations. It also helps students to maximize their credit hours and graduate on time. Schedule Planner also analyzes student preference data to predict the optimal number of course sections to offer and helps to evenly distribute class fill rates. It enables advisers to create course schedules for groups of students at a time. One of their main  goals is to allow students to focus on which courses to take rather than worrying about when they are being offered.
\subsubsection{Pathway Planner}
The Pathway Planner allows students to plan their schedules in a multi-year format to encourage seeing the bigger picture and to plan ahead. It provides visuals to show students how their predicted course loads will affect their graduation date. Administrators can also see the courses that students plan on taking before registration. This allows for the addition and elimination of courses to best fit student needs. 

\subsection{Coursicle}
The slogan of Coursicle is ``Course registration sucks but Coursicle makes it better." The features of Coursicle are: students can receive text or email notifications when a seat opens up in class, students can schedule their courses using an attractive schedule planner, students can search through courses more easily with a variety of filters, students can create schedules with all prospective classes and then narrow them down to one workable schedule, students can easily compare textbook prices online through Coursicle, and students can view what classes their classmates are signed up for via Facebook. 

\subsection{Individual Student Software}
There are other types of download-able software currently available for students to use individually. These systems are for individual use, and are not tailored for institutional implementation. To use these systems, students input information about their education, such as classes, credits, and requirements. This data is then used to create organized visualizations to help students to better see their goals and pathway. A popular generic system is the Microsoft Office College Credit Planner Template. Many individual colleges and universities have their own custom download-able course planning spreadsheets as well.

\section{Social Network}
\subsection{LinkedIn}
LinkedIn is widely known for being the world's largest professional network. It sets itself apart from other popular social media sites by being focused solely on building professional identities and forging professional relationships. There are six major components to LinkedIn: Home, Profile, My Network, Learning, Jobs, and Interests.
\subsubsection{Home}
A user's homepage is arranged in a feed type format, with quick information about your profile, profile views, and incoming messages. The feed section contains recent updates from connections and companies related to your interests. There are also sections that encourage engagement--for instance, quick ways to ``share an update", ``upload a photo", or ``write and article" and suggestions to ``reconnect with your colleagues" and to add someone you may know. 
\subsubsection{Profile}
A user's profile page is available for other LinkedIn users to see. Users can decide what information they would like to share about themselves, but it is all limited to professional related categories such as education, work experience, volunteer work, and skills and endorsements. 
\subsubsection{My Network}
A user's network includes current connections, recommended connections, connections added through outside contact information, and contacts added through an alumni network. 
\subsubsection{Learning}
LinkedIn offers online courses on professional development topics such as leadership, storytelling, creating alliances with employees, and winning back a lost customer. There are also field-related courses, such as online code courses. These courses are often in the form of videos, and can be accessed by premium LinkedIn members. 
\subsubsection{Jobs}
Jobs on LinkedIn automatically suggest jobs for users based off the information on their profile. Jobs can also be searched for using keywords such as job title, company, and location. Users can set preferences to refine their automatic suggestions.
\subsubsection{Interests}
In the Interests section, users can follow companies and groups based off their personal interests. There are also links to SlideShare and ProFinder, which offer services for creating professional presentations and hiring local freelancers, respectively.

\subsection{TechHui}
The TechHui page describes itself as being ``Hawaii's Technology Community." The TechHui site has ten main sections: Profile, Members, Events, Forum, Groups, Photos, Videos, Blogs, Directory, and Coders.
\subsubsection{Profile}
Each user has a profile page which contains information such as a name, profile picture, occupation, areas of interest, software language proficiency and interests, and recent activity.
\subsubsection{Members}
The members page lists all members, including a section at the top for featured members. Each member is listed by their name, with their profile picture and location. Through this page, users can communicate with other users by commenting on other user's profile pages.
\subsubsection{Events}
The events page lists upcoming events and featured events. The event snippets include an imagine, a name, a time and date, a location, the name of the organizer, the type of event, and a brief description of the event. Users can click on these snippets to go to an event page, which includes more detailed information and allows users to respond to events with ``will attend", ``might attend" and ``will not attend." 
\subsubsection{Forum}
The forum page includes a list of technology related categories, which can be clicked on to access a list of related forums. It also includes some featured forums at the top. Some of these categories include ``General Software Development", ``Java Software Development", ``Funding Technology Startups", ``Software Design Patterns", ``Tech Jobs", ``Tech Resumes", ``Web Design", ``Tech Humor" and more. Users can both start discussion forums and respond to other users' forums. 
\subsubsection{Groups}
There are many different groups listed on this page, including some featured groups. Each group snippet has an image, a name, the amount of members in the group, the date of the group's latest activity, and a brief description of the group. Users can click on these snippets to learn more about the group and to join the group as well. Once in the group, users can participate in commenting on the group wall and creating and responding to group discussion forums. 
\subsubsection{Photos}
On the Photos page, users can easily view all public photos uploaded by users (including profile pictures). Featured photos are included as well. Users can view these photos and comment on them as well. 
\subsubsection{Videos}
On the Videos page, users can easily view all public videos uploaded by users. Featured videos are included as well. Users can view these videos and comment on them as well. 
\subsubsection{Blogs}
This page displays a feed of all users' blog posts. Posts are also organized by featured posts, latest posts, most popular posts, and monthly archives. Users can click on these blog posts to read them in their entirety and can comment on them as well. 
\subsubsection{Directory}
This page includes a listing of technology related jobs in Hawaii, organized into 21 subcategories. Users can click on these listings to view more details about the jobs, and also to access external websites.
\subsubsection{Coders}
This page lists web startups that are writing code in Hawaii. The list contains just the names of the startups, which can be clicked on to learn more at the startup website.

\subsection{Rate My Professors}
Rate My Professors allows users to communicate and share content with each other by posting reviews of colleges and professors. Although users can create accounts, the reviews are listed as anonymous. Other users can provide feedback on reviews with either a thumbs up (user found this to be useful) or thumbs down (user did not find this to be useful). The site also contains site-generated blog posts and videos, but users cannot directly interact with these.

\subsection{Other Popular Social Networks}
Social networks have become extremely popular and there are too many of them to describe in detail here. The top fifteen most popular social networks as of September 2016 are Facebook, Instagram, YouTube, Twitter, LinkedIn, Pinterest, Google+, Tumblr, Reddit, VK, Flickr, Vine, Meetup, Ask.fm, and ClassMates. While most of these are not academically focused, they could potentially host an academic environment. Additionally, while RadGrad could be integrated directly into one of these existing social networks (i.e. become a Facebook application), creating a standalone application does not exclude members who do not have a Facebook or are not active on Facebook, it does not depend on the continuing popularity of Facebook, and I believe it may develop a stronger sense of brand. 

\section{Gamification}
Since it would be ineffective and senseless to discuss every existing video game, I conducted a brief informal survey of some current ICS undergraduates regarding their current favorite video game. They listed the following games: League of Legends, Monster Hunter, sports games (i.e. NBA2k7), Hearthstone, RimWorld, Geometry Dash, Overwatch and Kerbal Space Program. In this section I will discuss my current favorite video game and three of the popular video games according to the ICS students.
\subsection{Summoners War}
Summoners War is a mobile fantasy RPG with over 60 million players worldwide. It is based off a freemium model, with many players playing for free, and many other players playing with in-application purchases. Based off the iTunes Summoners War page, the basic premise of Summoners War is as follows: ``Jump into the Sky Arena, a world under battle over the vital resource: Mana Crystals! Summon over 900 different types of monsters to compete for victory in the Sky Arena! Assemble the greatest team of monsters for strategic victories!"
	As with many games, one of the interesting things about it is it's ability to motivate users into completing several, often tedious and unenjoyable actions, in order to achieve a virtual reward. One of the examples of this is the weekly Arena Rank. The Arena is where players can battle against other players, in an attempt to reach as high a rank as possible. There are different ranks based off the amount of victories and defeats the player has had: Beginner, Challenger, Fighter, Conqueror, Guardian, and Legend. The Arena is often very difficult, and in order to achieve a certain standing, the player must constantly battle others, and set up a solid defense that cannot be defeated by other players. In order to improve one's defense and offense team, a player must spend hours doing grueling tasks, such as gathering magical essences, gaining EXP points to level monsters, and collect ``runes" which can be strategically placed on each monster to improve certain stats. Doing these tasks can take up a significant amount of time and energy. Each week, a player is awarded a certain rank based off their performance in the Arena that previous week. This rank is manifested in the form of a small icon next to the player's name. When players see other players' icons, they are immediately informed of that person's standing in the game, and the player him/herself will be rewarded with feelings of pride and satisfaction.

\subsection{League Of Legends}
League of Legends is a multiplayer online battle arena (MOBA) type of video game and also follows a freemium business model. In this game, the player assumes the character of a summoner who controls a champion with unique abilities, and they battle with a team of other champions against another team of champions (either other live players or computer controlled). The main goal of the game is to destroy the opposing team's nexus, which is a structure at the middle of the team's base and is protected by defensive structures. At the start of each match, all champions start off weak, but they can increase in strength throughout the game by accumulating items and experience. Each match typically lasts from 20-60 minutes. There are three different game modes: Summoner's Rift, Twisted Treeline, and Howling Abyss. Each game mode is similar in that a team of players must work together to accomplish a terminal objective and a victory condition. Each mode also includes smaller intermediate objectives that can help teams to get closer to victory. 
	Opposed to Summoner's War, gold gathered during the match and items purchased with that gold only last for that match, and do not carry over to future matches. Each match begins with each player being more or less equal in terms of advantage, regardless of how much time or effort the player has put in beforehand. 
	However, the game does include other incentives to continue to win games and see personal development. Players get player experiences from playing matches on a single account. As their experience increases, they can ascend from level 1 to 30. Higher level players are given access to different maps, game modes, and additional abilities and features which give players a small boost in battle. 
\subsection{Hearthstone}
Hearthstone is a free to play online collectible card video game. It is turn based between two opponents, who use constructed decks of thirty cards, and a selected hero with a unique power. Players can attack the opponent using mana points. The main goal is to reduce the opponent's health to zero. If the player wins, they can earn in-game gold, new cards, or other in-game prizes. Players can use the gold or microtransactions to purchase new cards to improve their decks. There are several different game modes: casual and ranked matches, daily quests, and weekly challenges. Unlike many other popular collectible card games, Hearthstone does not allow players to trade cards. Instead, players can disenchant their unwanted cards into arcane dust, which can then be used to craft new cards of the player's choice. 
\subsection{Overwatch}
Overwatch is a team based multiplayer first person shooter (FPS). Each team has six players, and each player may select one predefined hero character. There are four classes of heroes: Offense, Defense, Tank, and Support. Each hero character has unique movements, attributes, and skills. As the team is being set up, the game will provide advice if the team is unbalanced. However, once the game starts, players can still switch characters after a death or after returning to their home base. The team of heroes work together to secure and defend certain control points and/or escort a payload across the map in a certain amount of point. As players continue to play matches, they can gain rewards that do not affect gameplay, such as character skins and poses. At the end of each match, a server-determined Play of the Game (PotG) is replayed for all players. This play is based off certain factors such as a high scoring moves or effective use of a skill. Up to four individual achievements per team are highlighted, and afterwards players can vote for one to promote. The player who wins the most votes get a reward of experience points. 
	As players gain experience points, they can earn a loot box, which provides certain in-game prizes and in-game currency. If players do not have enough experience points for a loot box, they also have the option to obtain one through a microtransaction. 
	The game supports several different gameplays such as tutorial and practice modes, casual matchmaking, weekly brawls, custom games, and competitive play. Casual matchmaking allows players to play alone or with friends, and are randomly matched against others with similar skill levels. The weekly brawl gameplay was inspired by Hearthstone, and features matches with unique rules, which change weekly. Custom games allow users to have private or public games and can edit different options for that specific match. Competitive mode allows players within a certain region and on a certain platform, to become ranked. This mode is run in 2.5 month seasons. Only players at level 25 or above can participate. Participants also much first play ten preliminary matches which will assign the player a skill rating from 1 to 5000, which is used to create ideal matches. Similarly to the arena battles in Summoners War, there are seven skill ranking tiers: Bronze, Silver, Gold, Platinum, Diamond, Master, and Grandmaster. Players can be demoted to a lower tier or promoted to a higher tier based on their performance. Each competitive win awards a player with in-game currency. Players will also get an additional award based on their final ranking at the end of the season. 
