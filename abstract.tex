\begin{abstract}

A casual analysis of the Hawaii technology community site, TechHui, suggests that over the past decade, recent alumni and current undergraduates of the Information and Computer Science (ICS) program at the University of Hawaii at Manoa (UHM) have experienced several problems with various academic, professional, and social aspects of their ICS experience. A baseline student survey conducted in Spring 2017 reveals current and more detailed student perceptions on the academic, professional, and social aspects of the ICS degree experience. These baseline results can be used in a future study to measure if RadGrad has had any effects on the students. Existing degree planning systems such as STAR, Starfish by Hobsons, Blackboard Planner and Coursicle fail to provide the specific support that an ICS student needs to create a complete and comprehensive degree plan. Existing academic social networks such as LinkedIn and TechHui fail to connect students closely with professors and alumni. Current popular video games suggest several gamification features that could encourage ICS students to achieve higher goals. A new system called RadGrad combines degree planning, social networking, and gamification in a way that aims to give ICS undergraduates the support they need to succeed and redefines what it means to have a successful degree experience.
\end{abstract}
